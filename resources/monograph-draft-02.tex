\documentclass[11pt]{article}
\usepackage[margin=1in]{geometry}
\usepackage{amsmath,amssymb,amsthm}
\usepackage{hyperref}
\usepackage{graphicx}
\usepackage{float}
\usepackage{caption}

\title{Paintings as Sheaves of Latent Physical Structure}
\author{Flyxion}
\date{\today}

\begin{document}
\maketitle

\begin{abstract}
We formalize a view of painted artwork as an information-dense spatial model specifying implicit physical constraints. A painting induces a sheaf of locally coherent physical interpretations over the image plane. Animation is modeled as global section selection followed by a functorial lift to dynamical evolution. Ambiguities in motion are shown to correspond to non-uniqueness of global sections, while stylistic consistency corresponds to sheaf coherence and rigidity of restriction morphisms. This model explains why generative animation systems recover physically plausible motion from artwork containing strong internal physics priors, even when the scenes are imagined rather than observed.
\end{abstract}

\section{Image Space as a Topological Domain}

Let $X$ be the 2-dimensional support of a painting, treated as a topological space whose open sets $U \subseteq X$ correspond to spatial regions on the canvas. Local visual data in $U$ (color, brush stroke orientation, shading gradients, anatomical cues, specular highlights) imply a set of compatible \emph{physical interpretations} of that region.

Define a presheaf
\[
F : \text{Open}(X)^{\mathrm{op}} \to \mathbf{Set}
\]
where $F(U)$ is the set of all local interpretations of $U$ consistent with:
\begin{itemize}
    \item implied lighting and reflectance models,
    \item locally plausible geometry and surface normals,
    \item partial motion priors encoded by pose and articulation,
    \item any painterly conventions functioning as perceptual physics.
\end{itemize}

For $U \subseteq V$, restriction maps
\[
\rho_{V,U} : F(V) \to F(U)
\]
project a larger patch’s interpretations down to subregions by retaining only compatible constraints.

\section{Sheaf Condition as Internal Consistency}

The painting exhibits \emph{physical coherence} when $F$ satisfies the sheaf axioms:

1. (\textbf{Locality}) If sections $s,t \in F(U)$ satisfy $\rho_{U,U_i}(s)=\rho_{U,U_i}(t)$ for all $U_i$ in an open cover of $U$, then $s=t$.

2. (\textbf{Gluing}) For an open cover $\{U_i\}$ of $U$, any compatible family of local sections
\[
s_i \in F(U_i), \quad \rho_{U_i,U_i \cap U_j}(s_i) = \rho_{U_j,U_i \cap U_j}(s_j)
\]
admits a global section $s \in F(U)$ with $\rho_{U,U_i}(s)=s_i$.

A painting is:

\begin{itemize}
    \item \textbf{Physically realizable} iff $F(X) \neq \varnothing$.
    \item \textbf{Deterministic in its implied physics} iff $|F(X)| = 1$.
    \item \textbf{Ambiguous} iff $|F(X)| > 1$ (multiple valid global worlds).
    \item \textbf{Contradictory} iff $F(X)=\varnothing$.
\end{itemize}

Ambiguous elements, such as forms without directional bias (e.g., a dinosaur lacking a privileged head–tail axis), correspond precisely to the case $|F(X)|>1$.

\section{Animation as Functorial Dynamics}

Let $\mathbf{Dyn}$ be a category whose objects are dynamical world models (state spaces endowed with admissible motion fields) and whose morphisms preserve physical continuity.

An animation model implicitly factors as:

\begin{enumerate}
    \item \textbf{Global section selection:} choose
    \[
    s \in F(X)
    \]
    a single coherent physical interpretation of the painting.
    \item \textbf{Temporal lift:} apply a functor
    \[
    T : \mathbf{Set} \to \mathbf{Dyn}
    \]
    whose action on $s$ produces a dynamical evolution $T(s)$ consistent with local constraints encoded by the sheaf.
\end{enumerate}

Thus, animation is not synthesis \emph{from nothing}, but prolongation of latent structural constraints into a time-evolving physics.

\section{Optical Priors as Discrete Rendering Operators}

Visual light cues induce structured compatibility conditions in $F$ analogous to rendering operators:

\[
\rho^{\text{light}}_{V,U} : F(V) \to F(U)
\]

These behave like discrete approximations of light transport:

\begin{itemize}
    \item specular elongation $\rightarrow$ convolution with anisotropic kernels,
    \item bloom and flare $\rightarrow$ radial falloff priors,
    \item atmospheric desaturation $\rightarrow$ depth-coupled color restriction,
    \item mirrored structure $\rightarrow$ local involutive symmetry priors.
\end{itemize}

A painted image whose optical restriction maps approximate these operators will be animated as if it had been ray-traced, even when the image is imaginative rather than physically observed.

\section{Ambiguity as Moduli of Global Sections}

Let
\[
\mathcal{M} = F(X)
\]
be the moduli space of admissible global interpretations. Local disambiguating cues act as constraints reducing $\mathcal{M}$ by defining subsheaves $F' \subseteq F$ such that:

\[
\mathcal{M'} = F'(X), \qquad |\mathcal{M'}| \le |\mathcal{M}|.
\]

A painting decision (e.g., a biased stance or gaze direction) decreases $|\mathcal{M}|$ without compromising higher-order aesthetic ambiguity elsewhere.

When $|\mathcal{M'}|=1$, motion becomes determinate. When $|\mathcal{M'}|>1$, dynamics resolve arbitrarily among valid sections, producing effects like directionally invertible gait.

\section{Conclusion}

A painting acts as:

\begin{itemize}
    \item a partially specified physical theory compressed into image space,
    \item whose consistency is measured by sheaf coherence,
    \item whose ambiguities are global section multiplicity,
    \item and whose animation is a functorial lift to dynamics.
\end{itemize}

Thus, generative animation is not inventing motion but unfolding the painter’s implicit physics.

\vfill
\begin{center}
\emph{Motion does not enter the painting.\\It leaves it.}
\end{center}

\section{Higher Sheaf Structure and Homotopical Ambiguity}

The space of valid global interpretations of a painting often carries \emph{higher-order ambiguity}, not merely multiplicity but deformation paths between interpretations.

To capture this, we promote the sheaf 
\[
F : \mathrm{Open}(X)^{\mathrm{op}} \to \mathbf{Set}
\]
to an \emph{$\infty$-sheaf}
\[
\mathcal{F} : \mathrm{Open}(X)^{\mathrm{op}} \to \mathbf{Gpd}_\infty,
\]
whose values are $\infty$-groupoids encoding not only distinct interpretations, but homotopies between them.

Under this refinement:

\begin{itemize}
    \item Individual world-interpretations are 0-simplices.
    \item Continuous deformations between interpretations (e.g., forward- vs backward-walking dinosaurs) are 1-simplices.
    \item Higher coherence conditions on interpretation shifts form $k$-simplices.
\end{itemize}

A painting with \emph{binary but distinct resolutions} has a discrete 0-truncated $\infty$-sheaf.

A painting whose ambiguities admit smooth deformation lies in a connected coalescent of $\pi_0(\mathcal{F}(X))$, but possesses nontrivial $\pi_1$, giving rise to homotopy classes of interpretations.

Animation now factors as:

\[
T : |\mathcal{F}(X)| \longrightarrow \mathbf{Dyn},
\]

where $|\mathcal{F}(X)|$ is the \emph{geometric realization} of interpretation space.  
The selected animation corresponds to choosing a path-connected component, then lifting along a contractible trajectory to a dynamical section.

Thus, ambiguous motion is not a pathological failure but the presence of a nontrivial homotopy class in interpretation space.

Interpretation stability correlates with homotopical rigidity:

\[
\mathcal{F}(X) \text{ is rigid } \iff \pi_k(\mathcal{F}(X)) = 0 \ \forall k \ge 1.
\]

This reframes animation errors as higher-sheaf invariants rather than reconstruction defects.

\section{Theorem: Existence and Uniqueness of Animated Continuations}

\begin{theorem}
Let \( \mathcal{F} \) be the \(\infty\)-sheaf of physical interpretations induced by a painting on canvas \(X\), and let \(T : |\mathcal{F}(X)| \to \mathbf{Dyn}\) be a functor assigning dynamical evolutions to realizable global sections. Then:

1. A physically coherent animation exists if and only if \(\pi_0(\mathcal{F}(X)) \neq \varnothing\).
2. The animation is unique up to homotopy if and only if \(\mathcal{F}(X)\) is 0-truncated and contractible.
3. The animation exhibits directional ambiguity if \(|\pi_0(\mathcal{F}(X))| > 1\).
4. The animation exhibits continuously deformable ambiguity if \(\pi_1(\mathcal{F}(X)) \neq 0\).
\end{theorem}

\begin{proof}

(1) By definition, an animation requires a global physical interpretation. This corresponds to a global section, which exists precisely when \(\mathcal{F}(X)\) is non-empty, i.e., \(\pi_0(\mathcal{F}(X)) \neq \varnothing\).

(2) If \(\mathcal{F}(X)\) is 0-truncated and contractible, then it has exactly one connected component and no higher homotopy. Thus any two sections are homotopic, implying a unique dynamical evolution up to deformation.

(3) If \(|\pi_0(\mathcal{F}(X))| > 1\), there exist multiple disconnected sections. Selection of section prior to animation produces non-equivalent motions, e.g., forward- vs backward-propagating gaits.

(4) If \(\pi_1(\mathcal{F}(X)) \neq 0\), interpretations lie in the same connected component but admit nontrivial loops. This corresponds to smoothly deformable ambiguity rather than discrete choice (e.g., limb articulation that can continuously reverse without discontinuity).

Hence the structure of observed animation variance corresponds exactly to the homotopy type of \(\mathcal{F}(X)\).
\end{proof}

\section{Painting Layers as Subsheaf Decomposition}

To guide animation toward intended motion, the artist constructs the painting in three layered constraint systems, each corresponding to a subsheaf of the total physical interpretation sheaf $\mathcal{F}$.

\subsection{Macro Directionality Subsheaf}

This subsheaf $\mathcal{F}_{\mathrm{dir}} \subseteq \mathcal{F}$ encodes forward-backward orientation. Local sections assign to each open $U$ a directed axis consistent with:
\begin{itemize}
    \item head--body polarity,
    \item gaze or snout vector,
    \item asymmetric mass distribution,
    \item dominant thrust line,
    \item staggered limb positions.
\end{itemize}

Restriction maps $\rho_{V,U}^{\mathrm{dir}}$ preserve directional bias. A single asymmetry (e.g., one limb advanced) collapses $|\pi_0(\mathcal{F}_{\mathrm{dir}}(X))| = 1$, eliminating directional ambiguity.

\subsection{Mechanical Constraints Subsheaf}

The subsheaf $\mathcal{F}_{\mathrm{mech}} \subseteq \mathcal{F}$ specifies admissible articulations. For open $U$, sections include:
\begin{itemize}
    \item joint hinge loci,
    \item muscle contraction axes,
    \item spinal flexion limits,
    \item ground contact normals,
    \item occlusion-derived depth order.
\end{itemize}

These enforce biological plausibility; restriction ensures no invalid deformations propagate.

\subsection{Light and Material Subsheaf}

The optics subsheaf $\mathcal{F}_{\mathrm{opt}} \subseteq \mathcal{F}$ models discrete light transport. Local sections over $U$ contain:
\begin{itemize}
    \item directional falloff gradients,
    \item specular anisotropy,
    \item interreflection zones,
    \item atmospheric attenuation,
    \item shadow consistency.
\end{itemize}

Restriction maps $\rho_{V,U}^{\mathrm{opt}}$ mimic rendering operators, enabling temporal extrapolation as if ray-traced.

The total sheaf is the fiber product $\mathcal{F} \simeq \mathcal{F}_{\mathrm{dir}} \times \mathcal{F}_{\mathrm{mech}} \times \mathcal{F}_{\mathrm{opt}}$ over compatible local data. Intentional motion arises when each subsheaf is rigid ($|\pi_0(\mathcal{F}_i(X))| = 1$, $\pi_k(\mathcal{F}_i(X)) = 0$ for $k \geq 1$) in relevant domains.

\section{Workflow as Iterative Subsheaf Refinement}

The painting process proceeds as:
\begin{enumerate}
    \item Initial expressive sketch defines base presheaf $F_0$.
    \item Identify ambiguous loci where $|\pi_0(F_0(U))| > 1$.
    \item Introduce local cues to define subsheaf $F' \subseteq F_0$ with reduced $|\pi_0(F'(X))|$.
    \item Iterate until desired rigidity.
\end{enumerate}

This preserves aesthetic ambiguity outside motion-critical regions.

\section{Crystal City Coastline}
\begin{figure}[h]
    \centering
    \includegraphics[width=0.85\linewidth]{Crystal-city.jpg}
    \caption{Crystal City Coastline}
\end{figure}

The composition establishes a high, distant point of view over ocean and cliffs, terminating in a crystalline urban structure that reflects light as a second luminous source. The painted gradients encode an implicit atmospheric scattering model, with decreasing chromatic saturation at depth and increased specular coherence near the water surface. Even when static, the image contains optical priors that act like a latent rendering equation: bloom intensity, reflection elongation, and shoreline caustics collectively imply how the scene will behave under temporal extension.

\section{Dinosaurs in Underbrush}
\begin{figure}[h]
    \centering
    \includegraphics[width=0.85\linewidth]{Dinosaurs-in-underbrush.jpg}
    \caption{Dinosaurs in Underbrush}
\end{figure}

This scene embeds motion at the level of occlusion logic and texture flow rather than explicit gesture. The layered vegetation forms a topological depth stack through which small dinosaur bodies interleave, producing a natural motion prior: alternating reveal and conceal along a low trajectory. The local patches support multiple compatible articulation maps, yet jointly enforce a coherent ground-contact constraint, ensuring that animated motion resolves into traversal rather than displacement or drift.

\section{Maiasaurs with Nest}
\begin{figure}[h]
    \centering
    \includegraphics[width=0.85\linewidth]{Maiasaurus.jpg}
    \caption{Maiasaurs with Nest}
\end{figure}

The arrangement of bodies and eggs defines a gravitational center and social orientation without specifying locomotion. Poses encode rest vectors rather than velocity vectors, producing what can be described sheaf-theoretically as a low-motion global section with nontrivial topology: a world that is physically realizable but nearly static under temporal functorial lift. The animation that follows does not invent caretaking behavior but unfolds it from implied skeletal balance and gaze alignment.

\section{White Horse in Motion}
\begin{figure}[h]
    \centering
    \includegraphics[width=0.85\linewidth]{Painting-of-a-horse.jpg}
    \caption{White Horse in Motion}
\end{figure}

Unlike other images in the sequence, this painting encodes a relatively low-entropy motion prior. The stride phase, center-of-mass positioning, muscle grouping, and chromatic motion blur together imply a unique forward vector and gait cycle. The optic field contains anisotropic streaking aligned with the axis of acceleration, functioning as a local operator enforcing directional determinism. When animated, this scene exhibits near-singular global section selection, resulting in minimal ambiguity in motion.

\section{Lavender Planet Geology}
\begin{figure}[h]
    \centering
    \includegraphics[width=0.85\linewidth]{Lavender-planet.jpg}
    \caption{Lavender Planet Geology}
\end{figure}

The planetary surface is structured through vertically stratified geometry and non-terrestrial color channels, implying a spectral atmosphere distinct from Earth’s visible scattering regime. The compression of scale relationships between formations acts as a depth manifold with consistent curvature bias, allowing zoom-type motion to unfold without violating metric continuity. Unlike animated surface motion, this scene primarily supports camera movement through a fixed but richly encoded spatial field.

\section{Dragon on Cliff}
\begin{figure}[h]
    \centering
    \includegraphics[width=0.85\linewidth]{Dragon-on-cliff.jpg}
    \caption{Dragon on Cliff}
\end{figure}

This painting pre-solves the problem of motion by embedding hinge locations, load distribution, and counter-forces within the silhouette itself. Wing posture, body torque, and talon placement form a solvable system of rotational vectors, each grounded in a shared balance equation. When animated, the dragon does not have motion added externally; rather, time reveals the only physically coherent solution already implied by joint articulation and torque anchors. The viewer’s directed gaze is structurally encoded by head orientation and visual axis alignment, yielding a rare case where the global section not only determines motion but also point of address.

\section{Examples of Sheaf Structure in Specific Motifs}

We now formalize the three proposed examples.

\subsection{The Dinosaur’s Ambiguous Facing}

Let $X$ contain a symmetric dinosaur form. Define open cover $\{U_h, U_b, U_l\}$ for head, body, limbs.

Local sections:
\begin{itemize}
    \item $s_h \in \mathcal{F}(U_h)$: two possible head orientations.
    \item $s_b \in \mathcal{F}(U_b)$: symmetric torso.
    \item $s_l \in \mathcal{F}(U_l)$: symmetric limbs.
\end{itemize}

Restriction maps allow gluing in two incompatible ways:
\[
\mathcal{F}(X) \simeq \{ \text{left-facing}, \text{right-facing} \}, \quad |\pi_0(\mathcal{F}(X))| = 2.
\]

Animation functor $T$ yields two non-homotopic gaits. Adding asymmetric neck curve defines subsheaf $\mathcal{F}'$ with $|\pi_0(\mathcal{F}'(X))| = 1$.

\subsection{The Horse Gait as Uniquely Determined Global Section}

For a horse in mid-gallop with staggered hooves, asymmetric mane flow, and ground contact bias:

Local sections over limb regions $U_i$ specify unique phase offsets. Overlaps enforce cyclic coordination via biomechanical priors.

The sheaf $\mathcal{F}_{\mathrm{gait}}(X)$ satisfies:
\[
\pi_0(\mathcal{F}_{\mathrm{gait}}(X)) \simeq \{\text{canonical gallop}\}, \quad \pi_k(\mathcal{F}_{\mathrm{gait}}(X)) = 0 \ \forall k \geq 1.
\]

Thus $T$ produces a unique, biologically rigid animation up to temporal scaling.

\subsection{Lens Flare as Rigid Optics Subsheaf}

Consider a bright light source with radial streaks. Cover with annuli $\{A_i\}$ centered on the source.

Local sections $s_i \in \mathcal{F}_{\mathrm{opt}}(A_i)$ encode:
\begin{itemize}
    \item angular intensity falloff,
    \item chromatic dispersion,
    \item aperture ghosting.
\end{itemize}

Restriction maps $\rho_{A_{i+1},A_i}^{\mathrm{opt}}$ enforce radial symmetry and decay, yielding:
\[
\mathcal{F}_{\mathrm{opt}}(X) \text{ contractible}, \quad \pi_k(\mathcal{F}_{\mathrm{opt}}(X)) = 0 \ \forall k.
\]

Temporal lift $T$ extrapolates flare motion rigidly, as if optically simulated.

\section{Glossary}

\begin{description}

\item[Canvas space \(X\)]  
The 2-dimensional image treated as a topological space. Open sets \(U \subseteq X\) represent spatial regions of the painting.

\item[Site \((\mathrm{Open}(X), J)\)]  
The category of open regions with inclusions as morphisms, equipped with the Grothendieck topology \(J\) of open covers, modeling locality of perceptual constraints.

\item[Presheaf \(F : \mathcal{C}^{\mathrm{op}} \to \mathbf{Set}\)]  
Assignment of each region \(U\) to a set \(F(U)\) of locally valid physical interpretations (lighting, geometry, motion priors), with restriction maps expressing compatibility on subregions.

\item[Sheaf condition]  
Local interpretations that agree on overlaps uniquely glue into a larger interpretation. Ensures global physical coherence of the painted world.

\item[Global section \(s \in F(X)\)]  
A full, self-consistent world interpretation of the entire painting. Animation selects one such section before propagating it in time.

\item[Moduli space \(\mathcal{M}_F = F(X)\)]  
The set of all global sections (coherent world interpretations). Its cardinality measures physical determinacy vs ambiguity.

\item[Ambiguity classes]  
\(|\mathcal{M}_F| = 0\) (inconsistent), \(1\) (fully determined), or \(>1\) (multiple valid interpretations, e.g., directionally ambiguous motion).

\item[\(\infty\)-sheaf \(\mathcal{F}\)]  
A sheaf valued in \(\infty\)-groupoids rather than sets, encoding not only distinct interpretations but continuous deformations between them.

\item[Homotopy groups \(\pi_k(\mathcal{F}(X))\)]  
\(\pi_0\): distinct coherent world interpretations.  
\(\pi_1\): loops of continuously deformable interpretations (e.g., reversible or undirected motion).  
\(\pi_k, k > 1\): higher coherence conditions and nested ambiguities.

\item[Dynamical category \(\mathbf{Dyn}\)]  
A category whose objects are time-evolving physical states (motion fields, articulated bodies, lighting in motion), and whose morphisms preserve dynamical coherence.

\item[Animation functor \(T : \mathbf{Gpd}_\infty \to \mathbf{Dyn}\)]  
Maps a chosen world interpretation into a temporally extended animation, preserving local constraints and homotopy equivalences.

\item[Restriction maps \(\rho_{V,U}\)]  
Constraint projections for \(U \subseteq V\), ensuring local compatibility of physical interpretations.

\item[Subsheaf refinement \(F' \subseteq F\)]  
Introducing painterly disambiguation (e.g., stronger pose direction, mass asymmetry, gaze cues) corresponds to restricting the sheaf to fewer compatible interpretations.

\item[Stability index \(\sigma(\mathcal{F})\)]  
The largest \(k\) such that \(\pi_k(\mathcal{F}(X)) = 0\).  
High \(\sigma\): rigid, determinate motion.  
Low \(\sigma\): structured but ambiguous motion.  
Undefined or negative: self-contradictory constraints.

\item[Optical/physical priors]  
Painterly cues that function like discretized physical operators (light falloff, reflection, scattering, articulation limits) and are treated by the model as constrained physical rules rather than decoration.

\item[Directional ambiguity]  
Occurs when orientation cues are underspecified, yielding \(|\pi_0| > 1\), enabling multiple valid motion continuations (e.g., dinosaurs that can walk both directions).

\item[Homotopy rigidity]  
When all higher homotopy groups vanish, implying a single stable interpretation with no deformable alternatives.

\item[Functorial unfolding]  
The core thesis: animation does not invent motion but transports pre-existing physical constraints in the painting into time via a structure-preserving functor.

\end{description}

\appendix
\section{Appendix: Categorical Foundations of Painted Worlds}

This appendix formalizes the categorical structure underlying the interpretation of a painting as a physical constraint system.

\subsection{Painted Regions as a Site}

Let \( (X, \tau) \) be the topological space of the canvas with its usual open-set topology.  
Define the \emph{site of perceptual regions}:

\[
\mathcal{C} = (\mathrm{Open}(X), J)
\]

where \( \mathrm{Open}(X) \) is the category whose:

- Objects are open regions \( U \subseteq X \),
- Morphisms are inclusions \( i : U \hookrightarrow V \),

and \( J \) is the Grothendieck topology generated by open covers.  
This site encodes the principle that perception is local and compositional.

\subsection{The Painting Sheaf as a Constraint Presheaf}

A painting induces a presheaf of physically valid interpretations:

\[
F : \mathcal{C}^{\mathrm{op}} \to \mathbf{Set}
\]

where \( F(U) \) is the set of all internally consistent physical interpretations of region \( U \).  
Restriction morphisms

\[
\rho_{V,U} : F(V) \to F(U)
\]

discard constraints outside \( U \) while preserving compatibility on overlaps.

\(F\) is a sheaf if it satisfies descent with respect to \(J\).  
This ensures:

- No local region asserts contradictions when overlapped,
- Local physical cues extend consistently to larger regions.

\subsection{Global Sections and the World-Model Moduli Space}

The space of physically coherent world-models encoded by the painting is the set of global sections:

\[
\Gamma(X,F) = F(X)
\]

Define the \emph{moduli space of interpretations}:

\[
\mathcal{M}_F = \Gamma(X,F)
\]

with cardinality giving a coarse classification:

\[
|\mathcal{M}_F| =
\begin{cases}
0 & \text{(contradictory painting)}\\
1 & \text{(uniquely determined physics)}\\
>1 & \text{(directionally or physically ambiguous)}
\end{cases}
\]

This moduli space is refined in higher settings below.

\subsection{Promotion to an \(\infty\)-Sheaf of Interpretations}

Ambiguity between interpretations often admits continuous deformation rather than discrete choice.  
To encode this, lift \(F\) to an \(\infty\)-sheaf:

\[
\mathcal{F} : \mathcal{C}^{\mathrm{op}} \to \mathbf{Gpd}_\infty
\]

such that:

- 0-simplices are local physical interpretations,
- 1-simplices are continuous deformations between interpretations,
- higher \(k\)-simplices express coherence of deformations.

The global interpretation space becomes the \emph{homotopy type}

\[
\mathcal{M}_{\mathcal{F}} := \mathcal{F}(X) \in \mathbf{Gpd}_\infty
\]

with invariants:

\[
\begin{aligned}
\pi_0(\mathcal{M}_{\mathcal{F}}) & : \text{distinct motion-consistent worlds}\\
\pi_1(\mathcal{M}_{\mathcal{F}}) & : \text{continuous but nontrivial motion ambiguity}\\
\pi_k(\mathcal{M}_{\mathcal{F}}) & : \text{higher deformation coherence constraints}
\end{aligned}
\]

\subsection{Animation as a Derived Functor on Interpretation Space}

Let \( \mathbf{Dyn} \) be a category of physically admissible dynamical systems, where objects are state spaces with flow fields and morphisms preserve temporal continuity.

Animation is modeled as a derived functor

\[
T : \mathbf{Gpd}_\infty \to \mathbf{Dyn}
\]

such that:

1. \(T\) respects equivalences: homotopic interpretations animate to dynamically homotopic trajectories.
2. \(T\) preserves local constraints: optical and mechanical priors encoded in \(\rho_{V,U}\) lift to temporally stable behaviors.
3. Choice of connected component determines motion class:

   \[
   T\Big|_{\pi_0(\mathcal{M}_{\mathcal{F}})} : \pi_0(\mathcal{M}_{\mathcal{F}}) \to \pi_0(\mathbf{Dyn})
   \]

Ambiguous motion arises when multiple components exist; smoothly reversible or oscillatory motion corresponds to nontrivial \(\pi_1\).

\subsection{Constraint Tightening as Subsheaf Selection}

Painterly disambiguation imposes new local constraints, corresponding to subsheaf refinement:

\[
F' \subseteq F \quad \text{or} \quad \mathcal{F}' \hookrightarrow \mathcal{F}
\]

which reduces the moduli space:

\[
\mathcal{M}_{\mathcal{F}'} \hookrightarrow \mathcal{M}_{\mathcal{F}}
\]

Extreme cases:

- Adding decisive orientation cues collapses \(|\pi_0|\) to 1.
- Removing contradictory cues enforces non-emptiness.
- Preserving controlled ambiguity permits nontrivial \(\pi_1\) while eliminating higher instability.

\subsection{Interpretational Stability}

Define the \emph{stability index} of a painting as:

\[
\sigma(\mathcal{F}) = \sup \{ k \mid \pi_k(\mathcal{M}_{\mathcal{F}}) = 0 \}
\]

Then:

- \( \sigma = \infty \): completely rigid world-model
- \( \sigma = 0 \): discrete but stable alternative interpretations
- \( \sigma = 1 \): fluid but coherent motion ambiguity
- \( \sigma < 0 \): contradictory or non-animatable painting

\subsection{Conclusion}

A painted work encodes a finite-complexity \(\infty\)-sheaf whose homotopy invariants determine:

- whether animation exists,
- whether motion is unique, reversible, or multivalent,
- and whether ambiguities resolve discretely or continuously.

Animation is therefore not generative hallucination but \emph{functorial unfolding of implicit homotopical physics embedded in paint}.

\begin{thebibliography}{99}

\bibitem{maclane1998}
Mac Lane, Saunders.
\newblock {\em Categories for the Working Mathematician}.
\newblock Springer, 1998.

\bibitem{artin2011}
Artin, Emil and others.
\newblock {\em Algebraic Topology}.
\newblock Foundations for sheaf and homotopy concepts, 2011.

\bibitem{gelfand2003}
Gelfand, Sergei and Manin, Yuri.
\newblock {\em Methods of Homological Algebra}.
\newblock Springer, 2003.

\bibitem{ghrist2014}
Ghrist, Robert.
\newblock Elementary Applied Topology.
\newblock Applied sheaf theory and sensors, relevant to constraint-topology interpretation, 2014.

\bibitem{cohen2022}
Cohen, T. and Freeman, W.
\newblock Neural Rendering and the Physical Priors of Vision.
\newblock {\em Neural Computation}, 2022.
\newblock Explains learned optical invariants in generative models.

\bibitem{efros2020}
Efros, A.
\newblock Recognizing Image Structure Without 3D.
\newblock {\em Foundations of Computer Vision}, 2020.
\newblock Perceptual cues as structural priors.

\bibitem{hartshorne1977}
Hartshorne, Robin.
\newblock {\em Algebraic Geometry}.
\newblock Springer, 1977.
\newblock Canonical reference for sheaf-theoretic foundations.

\bibitem{baradad2021}
Baradad, M. and Isola, P.
\newblock Emergent Physical Priors in Deep Vision Models.
\newblock {\em NeurIPS}, 2021.
\newblock Learnt physics-like priors in image understanding.

\end{thebibliography}

\end{document}
